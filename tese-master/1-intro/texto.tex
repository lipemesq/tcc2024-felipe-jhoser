\chapter{Introdução}

%=====================================================

% A introdução geral do documento pode ser apresentada através das seguintes seções: Desafio, Motivação, Proposta, Contribuição e Organização do documento (especificando o que será tratado em cada um dos capítulos). O Capítulo 1 não contém subseções\footnote{Ver o Capítulo \ref{cap-exemplos} para comentários e exemplos de subseções.}.

O estudo das alianças em grafos é um campo fascinante que combina conceitos de teoria dos grafos e complexidade computacional. As alianças, especialmente as alianças defensivas, são subconjuntos de vértices que garantem uma forma de defesa mútua entre seus membros; conceito este que pode ser aplicado desde alianças para suporte mútuo entre nações em guerra \cite{kristiansen2004alliances} até análise da estrutura secundária do RNA \cite{Haynes2006}.

Neste estudo, abordamos o problema de encontrar alianças defensivas em grafos, que pode ser formalizado como a identificação de subconjuntos de vértices que satisfazem condições específicas de conectividade e vizinhança. Um grafo $G = (V, E)$ é composto por um conjunto de vértices $V$ e um conjunto de arestas $E$. Uma aliança defensiva é definida como um subconjunto $S \subseteq V$ tal que, para cada vértice $v \in S$, o número de vizinhos de $v$ dentro de $S$ é pelo menos igual ao número de vizinhos de $v$ fora de $S$. Essa propriedade assegura que cada vértice na aliança possui mais aliados do que potenciais inimigos, promovendo assim a segurança do grupo.

A complexidade computacional associada à identificação de alianças defensivas é desanimadora, e fazemos dela o ponto central deste estudo. Através da perspectiva atenuante do algoritmo FPT proposto por \cite{Enciso2009}, buscamos entender a eficiência e a viabilidade de encontrar tais alianças em grafos de diferentes tamanhos e estruturas.

Além disso, apresentamos um visualizador web que ilustra o funcionamento do algoritmo, permitindo uma compreensão mais intuitiva dos passos envolvidos na busca por alianças defensivas, propomos duas melhorias para a eficiência do algoritmo e disponibilizamos em um repositório o algoritmo final e otimizado em \textit{Python}. A seguir, detalharemos a fundamentação teórica necessária para a compreensão do tema, além de descrever a metodologia utilizada e os resultados obtidos.

%=====================================================
