\begin{resumo}

Este trabalho explora o problema de encontrar alianças defensivas em grafos, um tema relevante na teoria dos grafos com aplicações em diversas áreas, desde análise de mercado, redes sociais e biologia. Definimos uma aliança defensiva como um subconjunto de vértices onde cada vértice possui pelo menos tantos vizinhos dentro do conjunto quanto fora dele, de forma a ter sempre mais "aliados" do que "inimigos". O texto se concentra na formulação do problema, na análise da complexidade computacional e na implementação de um algoritmo eficiente para a identificação dessas alianças.

Apresentamos um algoritmo FPT semelhante a um busca em profundidade, que explora sistematicamente os vértices do grafo para encontrar alianças defensivas de tamanho específico, juntamente com duas novas melhorias que apresentam melhora significativa no desempenho, e um visualizador web desenvolvido que permite a visualização passo a passo do processo de busca.

\end{resumo}
